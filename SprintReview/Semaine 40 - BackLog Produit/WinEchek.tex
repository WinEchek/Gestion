\documentclass{article}
\usepackage[top=1cm, bottom=1.5cm, left=1.5cm, right=1.5cm]{geometry}
\usepackage[latin1]{inputenc}
\usepackage[T1]{fontenc}
\usepackage[francais]{babel}
\usepackage{lmodern}
\usepackage{graphicx}
\usepackage{xcolor}

\title{%
    \begin{minipage}\linewidth
        \centering \bfseries
        WinEchek
        \vskip3pt
        \large Un jeu qu'il est bien pour jouer aux echek
    \end{minipage}
}

\date{}

\definecolor{green}{rgb}{0,0.6,0}

\begin{document}

% d�finition des style de puces
\renewcommand{\labelitemi}{$\bullet$}
\renewcommand{\labelitemii}{$\circ$}
\renewcommand{\labelitemiii}{$-$}
\renewcommand{\labelitemiv}{$\triangleright$}

\maketitle

\section{Backlog produit}
\begin{itemize}
 \item Gestion de l'affichage et des contr�lse de l'utilisateur (IHM)
 \begin{itemize}
  \item{\textcolor{red}{Haute}} L'utilisateur pourra lancer une partie contre un adversaire de type humains sur la m�me machine
  \item{\textcolor{red}{Haute}} L'utilisateur pourra lancer une partie contre un adversaire en local
  \item{\textcolor{orange}{Moyenne}} L'utilisateur pourra lancer une partie contre un adversaire en r�seau
  \item{\textcolor{green}{Basse}} L'utilisateur pourra lancer une partie contre une intelligence artificielle
  \item{\textcolor{red}{Haute}} Pr�sence du plateau et des pi�ces lors de l'affichage d'une partie
  \item{\textcolor{orange}{Moyenne}} Pr�sence d'un timer dans l'affichage d'une partie
  \item{\textcolor{orange}{Moyenne}} Pr�sence d'un historique des coups dans l'affichage d'une partie
  \item{\textcolor{red}{Haute}} Pr�sence d'un affichage des pi�ces prises lors d'une partie
  \item{\textcolor{red}{Haute}} L'utilisateur pourra contr�ler le d�placement de ses pi�ces gr�ce � la souris
  \item{\textcolor{orange}{Moyenne}} Afficher les coups possibles lorsque le joueur s�lectionne une pi�ce
  \item{\textcolor{green}{Basse}} Afficher un menu pour modifier des options d'affichage ou sauvegarder un historique de partie
  \item{\textcolor{green}{Basse}} Sauvegarder une partie en cours
 \end{itemize}
 \item Moteur de jeu et contr�le des parties
 \begin{itemize}
  \item{\textcolor{red}{Haute}} D�roulement en tour par tout
  \item{\textcolor{red}{Haute}} Contr�le des d�placements possibles
  \item{\textcolor{red}{Haute}} Contr�le des positions d'�chec
  \item{\textcolor{red}{Haute}} Contr�le de prise des pi�ces
  \item{\textcolor{red}{Haute}} Contr�le de la promotion d'un pion en une autre pi�ce prise par l'adversaire
  \item{\textcolor{red}{Haute}} Contr�le de fin de partie (�chec et pat)
  \item{\textcolor{red}{Haute}} Contr�le de fin de partie (�chec et mat)
 \end{itemize}
 \item Gestion des parties en r�seau
 \begin{itemize}
  \item{\textcolor{red}{Haute}} Permettre � l'utilisateur de jouer en r�seau avec quelqu'un d'autre soit en local (sur un autre ordinateur)
  \item{\textcolor{orange}{Moyenne}} Permettre � l'utilisateur de jouer en r�seau avec quelqu'un d'autre � travers internet
 \end{itemize}
 \item Mise en place d'un joueur virtuel
 \begin{itemize}
  \item{\textcolor{green}{Basse}} L'utilisateur jouera contre un joueur virtuel (intelligence artificielle) qui proposera plusieurs niveaux de difficult�
 \end{itemize}
\end{itemize}


\section{A faire}
\begin{itemize}
 \item Dossier de conception
 \item Documents � rendre � M. Aubert
\end{itemize}

\end{document}