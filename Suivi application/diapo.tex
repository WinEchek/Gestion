\documentclass[usepdftitlre=false, debug]{beamer}

\usepackage[francais]{babel}
\usepackage[T1]{fontenc}
\usepackage[ansinew]{inputenc}
\usepackage{lmodern}
\usepackage{graphicx}
\usepackage{listings}
\usepackage{color}
\usepackage{pgf}
\usepackage{tikz}
\usetikzlibrary{arrows,automata}



%%%%%%%%%%%%%%%%%%%%%%%%%%%%%%%%%%%%%%%%%%%%%%%%%%%%%%%%%%%%%%%%%%%%%%%%%%%%%%%%%%%%%%%%%%%%%%%%
\usetheme{Rochester}
\usecolortheme{default}

\title{WinEchek}
\author{Mathis Deloge, Antoine Petot, Ange Picard, Arthur Carchi, Lucas Fougerouse, Vincent Dereclenne}
\institute{IUT Informatique Dijon / Auxerre}
\date{Lundi 14 Novembre 2016}
%%%%%%%%%%%%%%%%%%%%%%%%%%%%%%%%%%%%%%%%%%%%%%%%%%%%%%%%%%%%%%%%%%%%%%%%%%%%%%%%%%%%%%%%%%%%%%%%



\definecolor{mygreen}{rgb}{0,0.6,0}
\definecolor{mygray}{rgb}{0.5,0.5,0.5}
\definecolor{mymauve}{rgb}{1,0,0}

\lstset{ %
  backgroundcolor=\color{gray!30!white},   % choose the background color
  basicstyle=\small\ttfamily,        % size of fonts used for the code
  breaklines=true,                 % automatic line breaking only at whitespace
  captionpos=b,                    % sets the caption-position to bottom
  commentstyle=\color{mygreen},    % comment style
  escapeinside={\%*}{*)},          % if you want to add LaTeX within your code
  keywordstyle=\color{blue},       % keyword style
  stringstyle=\color{mymauve},     % string literal style
	numbers=left,
	frame=leftline,
	xleftmargin=42pt
}

\setbeamertemplate{navigation symbols}{%
\insertbackfindforwardnavigationsymbol
}

\setbeamercolor{background canvas}{bg=yellow!10!white}

\AtBeginSubsection[]
{
  \begin{frame}
  \frametitle{Sommaire}
  \small \tableofcontents[currentsection, currentsubsection]
  \end{frame}
}

\begin{document}

\begin{frame}
	\titlepage
\end{frame}

\section{WinEchek}
\begin{frame}
	\frametitle{WinEchek}
	\begin{block}{Rappel des enjeux}
	D�veloppement d'une applicaiton de jeu d'�chec afin de permettre une mont�e en connaissance, notament avec le langage C\#.
	\end{block}

\end{frame}

\begin{frame}
	\frametitle{WinEchek}
	\begin{block}{Description du projet}
	Un jeu d'�chec permettant de :
	\begin{itemize}
	 \item Jouer en local contre un autre joueur
	 \item Jouer en local contre une intelligence artificielle
	 \item Jouer en r�seau contre un autre joueur
	\end{itemize}

	\end{block}
\end{frame}


\section{Menu}
\begin{frame}
	\frametitle{Menu}
	\begin{block}{Un menu principal}
		\begin{itemize}
			\item Jouer en local : Pour jouer une partie locale contre un autre joueur
			\item Jouer en r�seau : Pour jouer une partie en r�seau
			\item Jouer contre l'ordinateur : Pour jouer une partie contre un ordinateur
			\item Param�tres : Permet d'acc�der aux param�tres de WinEchek
			\item Contribuer : Pour participer au d�veloppement de WinEchek
		\end{itemize}
	\end{block}
	\begin{center}
	 \includegraphics[width=10cm]{Images/MenuPrincipal.jpg}
	\end{center}
\end{frame}

\subsection{Jouer en local}
\begin{frame}
	\frametitle{Jouer en local}
	\begin{block}{Un sous menu}
		\begin{itemize}
		 \item Cr�er une nouvelle partie
		 \item Reprendre une partie
		\end{itemize}
	\end{block}
	\begin{center}
	 \includegraphics[width=9cm]{Images/JouerEnLocal.jpg}
	\end{center}
\end{frame}


\section{Interface graphique}
\begin{frame}
	\frametitle{Interface graphique}
	\begin{block}{GUI}
		\begin{itemize}
		 \item Application responsive
		 \item Couleur de l'application
		 \item Menu des options de partie
		 \item Historique des coups
		 \item Acc�s au git
		 \item Affichage d'un plateau de jeu
		 \item Affichage de pi�ce sur le plateau de jeu
		\end{itemize}
	\end{block}
\end{frame}

\subsection{Application responsive}
\begin{frame}
	\frametitle{Une application enti�rement responsive}
	\begin{center}
	 \includegraphics[width=10cm]{Images/responsive1.jpg}
	\end{center}
\end{frame}

\begin{frame}
	\frametitle{Une application enti�rement responsive}
	\begin{center}
	 \includegraphics[width=8cm]{Images/responsive3.jpg}
	\end{center}
\end{frame}

\begin{frame}
	\frametitle{Une application enti�rement responsive}
	\begin{center}
	 \includegraphics[width=4cm]{Images/responsive2.jpg}
	\end{center}
\end{frame}

\subsection{Couleur de l'application}
\begin{frame}
	\frametitle{Couleur de l'application}
	\begin{center}
	 \includegraphics[width=10cm]{Images/OptionsCouleurs.jpg}
	\end{center}
\end{frame}

\subsection{Menu des options de partie}
\begin{frame}
	\frametitle{Menu des options de partie}
	\begin{center}
	 \includegraphics[width=3cm]{Images/Options.jpg}
	\end{center}
\end{frame}

\subsection{Historique des coups}
\begin{frame}
	\frametitle{Historique des coups}
	\begin{center}
	 \includegraphics[width=4cm]{Images/Historique.jpg}
	\end{center}
\end{frame}

\subsection{Acc�s au git}
\begin{frame}
	\frametitle{Acc�s au git}
	\begin{center}
	 \includegraphics[width=7cm]{Images/git.jpg}
	\end{center}
\end{frame}

\subsection{Affichage du plateau et des pi�ces}
\begin{frame}
	\frametitle{Affichage du plateau et des pi�ces}
	\begin{center}
	 \includegraphics[width=11cm]{Images/board1.jpg}
	\end{center}
\end{frame}


\section{Logique de jeu}
\subsection{D�placer une pi�ce}
\begin{frame}
	\frametitle{D�placer une pi�ce}
	\begin{center}
	 \includegraphics[width=11cm]{Images/board1.jpg}
	\end{center}
\end{frame}

\begin{frame}
	\frametitle{D�placer une pi�ce}
	\begin{center}
	 \includegraphics[width=11cm]{Images/board2.jpg}
	\end{center}
\end{frame}

\begin{frame}
	\frametitle{D�placer une pi�ce}
	\begin{center}
	 \includegraphics[width=11cm]{Images/board3.jpg}
	\end{center}
\end{frame}

\subsection{Prendre une pi�ce}
\begin{frame}
	\frametitle{Prendre une pi�ce}
	\begin{center}
	 \includegraphics[width=11cm]{Images/board1.jpg}
	\end{center}
\end{frame}

\begin{frame}
	\frametitle{Prendre une pi�ce}
	\begin{center}
	 \includegraphics[width=11cm]{Images/board2.jpg}
	\end{center}
\end{frame}

\begin{frame}
	\frametitle{Prendre une pi�ce}
	\begin{center}
	 \includegraphics[width=11cm]{Images/board4.jpg}
	\end{center}
\end{frame}

\subsection{Annuler / Rejouer un coup}
\begin{frame}
	\frametitle{Annuler / Rejouer un coup}
	\begin{center}
	 \includegraphics[width=11cm]{Images/board5.jpg}
	\end{center}
\end{frame}

\begin{frame}
	\frametitle{Annuler / Rejouer un coup}
	\begin{center}
	 \includegraphics[width=11cm]{Images/board6.jpg}
	\end{center}
\end{frame}

\begin{frame}
	\frametitle{Annuler / Rejouer un coup}
	\begin{center}
	 \includegraphics[width=11cm]{Images/board7.jpg}
	\end{center}
\end{frame}

\begin{frame}
	\frametitle{Annuler / Rejouer un coup}
	\begin{center}
	 \includegraphics[width=11cm]{Images/board8.jpg}
	\end{center}
\end{frame}

\subsection{Sauvegarder / Charger}
\begin{frame}
	\frametitle{Sauvegarder / Charger}
	\begin{block}{Options}
		\begin{itemize}
		 \item Sauvegarder : Pour sauvegarder une partie et la reprendre plus tard
		 \item Charger : Pour charger une partie pr�c�demment sauvegarder.
		\end{itemize}
	\end{block}
	\begin{center}
	 \includegraphics[width=2cm]{Images/Options.jpg}
	\end{center}
\end{frame}

\begin{frame}
	\frametitle{Sauvegarder / Charger}
	\begin{center}
	 \includegraphics[width=11cm]{Images/sauvegarde.jpg}
	\end{center}
\end{frame}

\begin{frame}
	\frametitle{Sauvegarder / Charger}
	\begin{center}
	 \includegraphics[width=11cm]{Images/charger.jpg}
	\end{center}
\end{frame}

\subsection{Quitter une partie}
\begin{frame}
	\frametitle{Quitter une partie}
	\begin{center}
	 \includegraphics[width=11cm]{Images/quitter.jpg}
	\end{center}
\end{frame}


\section{Pour la suite...}
\begin{frame}
	\frametitle{Pour la suite...}
	\begin{center}
	D�finissons ensemble le prochain sprint !
	\end{center}
\end{frame}

\begin{frame}
  \frametitle{Sommaire}
  \tableofcontents
\end{frame}

\end{document}