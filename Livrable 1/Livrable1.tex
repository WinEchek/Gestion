\documentclass{article}
\usepackage[top=1cm, bottom=1.5cm, left=1.5cm, right=1.5cm]{geometry}
\usepackage[ansinew]{inputenc}
\usepackage[T1]{fontenc}
\usepackage[french]{babel}
\usepackage{lmodern}
\usepackage{graphicx}
\usepackage{listings}
\usepackage{color}
\usepackage{pgf}
\usepackage{tikz}
\usepackage{url}
\usepackage{graphicx}
\usetikzlibrary{arrows,automata}


\definecolor{mygreen}{rgb}{0,0.6,0}
\definecolor{mygray}{rgb}{0.5,0.5,0.5}
\definecolor{mymauve}{rgb}{1,0,0}


\lstset{ %
  backgroundcolor=\color{white},   % choose the background color
  basicstyle=\footnotesize,        % size of fonts used for the code
  breaklines=true,                 % automatic line breaking only at whitespace
  captionpos=b,                    % sets the caption-position to bottom
  commentstyle=\color{mygreen},    % comment style
  escapeinside={\%*}{*)},          % if you want to add LaTeX within your code
  keywordstyle=\color{blue},       % keyword style
  stringstyle=\color{mymauve},     % string literal style
	numbers=left,
	frame=leftline,
	xleftmargin=42pt
}

\title{%
    \begin{minipage}\linewidth
        \centering \bfseries
	  WinEchek
        \vskip3pt
        \large Dossier préliminaire
    \end{minipage}
}

\author{Mathis DELOGE, Antoine PETOT, Ange PICARD\\Arthur CARCHI, Lucas FOUGEROUSE, Vincent DERECLENNE}
\date{}


\begin{document}

% définition des style de puces
\renewcommand{\labelitemi}{$\bullet$}
\renewcommand{\labelitemii}{$\circ$}
\renewcommand{\labelitemiii}{$-$}
\renewcommand{\labelitemiv}{$\triangleright$}


\maketitle

\section{Introduction}
\subsection{Besoins}
WinEchek est une application Windows qui permettra de jouer aux échecs sur un ordinateur. Les parties pourront être dispensées de différentes manières. La première sera de jouer à deux joueurs humains sur le même ordinateur. La seconde permettra à deux joueurs humains de jouer en réseau (local) sur la même partie. Enfin, la dernière proposera à un joueur seul de dispenser une partie contre l'ordinateur qui sera doté d'une intelligence artificielle.

\subsection{Fonctionnalités principales}
\begin{itemize}
 \item Jouer une partie
 \begin{itemize}
  \item Contre un adversaire
  \item Contre une IA
  \item Via le réseau
  \item Bouger une pièce
  \item Voir l'historique des coups
  \item Voir les coups possibles
 \end{itemize}
 \item Sélectionner niveau IA
 \item Enregistrer une partie
 \item Reprendre une partie enregistrée
\end{itemize}

\subsection{Objectif}
Le but du projet n'est pas forcément d'implémenter le plus de fonctionnalités possibles, ni même de proposer quelque chose d'innovant. En effet, ce type d'application étant assez difficile à réaliser, les projets existants seront forcément plus intéressants que notre réalisation finale.
Notre objectif est donc de mettre en pratique, au travers de la réalisation d'une application complexe, les différents domaines abordés dans notre formation (IHM, CO, POO, Réseau) mais également de découvrir les bases de la programmation d'intelligences artificielles.
Pour résumer, nous réalisons cette application plus dans un but pédagogique que pour combler un besoin (les solutions existantes sont très performantes, et la plupart proposent déjà nos fonctionnalités).

\section{Fonctionnalités principales}

cf BackLog produit

\newpage
\section{Existant}
\subsection{Moteurs}

Les moteurs d'échecs sont le cœur des jeux d'échecs. Ce sont eux qui gèrent toutes les contraintes des parties d'échecs et ont un module d'intelligence artificielle intégré.

\subsubsection{StockFish}
\url{https://stockfishchess.org/}
\newline

Stockfish est un programme d'échecs open source développé par Tord Romstad, Marco Costalba, Joona Kiiski et Gary Linscott. Ce programme est libre et gratuit et est le meilleur logiciel d'échecs non commercial au monde depuis mai 2014. Sur l'ensemble des logiciels, il est considéré comme un des meilleurs avec Komodo 9.3 et Houdini 4. 

\subsubsection{Komodo}
\url{https://komodochess.com/}
\newline

Komodo est un programme d'échecs commercial et non libre créé par Don Dailey, Mark Lefler et Larry Kaufman en 2007. Il a été considéré comme le meilleur programme d'échecs en 2016.

\subsubsection{GNU Chess}
\url{https://www.gnu.org/software/chess/}
\newline

GNU Chess est un logiciel libre de jeu d'échecs, sous les termes de la licence publique générale GNU, maintenu par la collaboration de développeurs. Ne disposant que d'une saisie des coups en ligne de commande, il peut être considéré comme un moteur d'échecs. Il est souvent utilisé avec un environnement graphique comme XBoard ou GlChess pour la 3D.

\subsection{Deep Blue}
\url{https://fr.wikipedia.org/wiki/Deep_Blue}
\newline

Deep Blue est un superordinateur conçu par IBM destiné à jouer aux échecs. Il a battu Garry Kasparov, un des meilleurs joueurs d'échecs au monde, en 1997. Ce fut un événement remarquable dans le monde de l'intelligence artificielle, car dès lors, aucun ordinateur n'avait réussi à battre un joueur humain de si haut niveau aux échecs, et on pensait que ce ne serait pas possible avant un long moment.

\section{Solutions similaires}
\subsection{WJChess 3d}
\url{https://fr.jeffprod.com/wjchess/}
\newline

WJChess 3d est jeu d'échecs pour PC fonctionnant sous Windows. Il inclut de nombreuses options et fonctionnalités et possède la particularité d'avoir une partie graphique assez élaborée et modifiable à volonté.
Ce logiciel ne permet qu'un mode 2 joueur ou contre l'ordinateur, il intègre les possibilités de sauvegarder, reprendre une partie ou même créer une position grâce à l'importation / exportation des positions au format FEN.
De plus, il respecte toutes les règles des échecs et propose plusieurs niveaux de difficulté avec un temps de recherche ou une profondeur fixe pour l'intelligence artificielle.

\subsection{Jeux en ligne}
Il existe de nombreux jeux d'échecs en ligne et beaucoup ne proposent que quelques fonctionnalités très simples.

\paragraph{Jouer-aux-echecs.fr}
\url{http://jouer-aux-echecs.fr/}
\newline

Ce jeu d'échec en ligne ne possède que quelques fonctionnalités parmi lesquelles nous pouvons trouver le choix de la couleur de nos pions, du niveau de jeu du logiciel Shredder mais nous avons également la possibilité d'annuler un tour. Concernant les graphismes, c'est une représentation très simpliste, en 2D avec une coloration des cases en bleu pour savoir à chaque tour quelles sont nos possibilités de mouvements.

\paragraph{Jeu d'échecs flash}
\url{http://www.stratozor.com/echecs/jeu-echecs-flash.php}

C'est un jeu d'échecs en ligne basé sur la technologie Flash. Il permet entre autres d'afficher l'historique des coups et les mouvements possibles en cliquant sur une pièce. 

\subsection{Lichess}
\url{https://fr.lichess.org/}
\newline

Lichess est un jeu d'échecs par navigateur qui permet de faire des parties contre d'autres joueurs réels, une fois connecté sur le site, ou bien contre une IA avec différents niveaux de difficulté qui vont de 1 à 10. Le site propose aussi des tutoriels d'initiation aux échecs et différents entraînements en fonction de son niveau de maîtrise.



\newpage
\section{Technologies}
\subsection{Préambule}

Concernant le type d'application, nous avons opté pour une application de bureau. Le cycle de développement ainsi que les technologies nous correspondant plus que celles utilisées pour une application web. En ce qui concerne les parties entre deux joueurs humains en ligne, pour des raisons de budget et de flexibilité, nous avons fait le choix d'utiliser une architecture client-serveur / client plutôt qu'une architecture client / serveur / client, c'est-à-dire que ce ne sera pas un serveur distinct qui hébergera la partie, mais un des deux clients, cela nous permet de ne pas avoir à matérialiser un serveur et nous donnera l'occasion de découvrir l'utilisation de cette architecture.

\subsection{Langage}

Pour ce qui est du langage utilisé pour l'implémentation, nous avons évalué trois options, le C++, le C\# et Java.
\newline

Le principal atout du C++ est la performance, en effet celui-ci permet de descendre plus bas niveau que le C\# ou le Java. D'autre part, la bibliothèque standard est moins haut niveau. Les outils fournis en comparaison au Java/C\# ne permettent pas un développement rapide ceux-ci étant plus performants mais moins simple d'utilisation, ce qui, dans le cadre de notre projet, est peu pertinent puisque nous ne cherchons pas à réaliser un programme performant. De plus, il nous aurait fallu utiliser un framework comme Qt pour réaliser la partie graphique, et l'environnement de développement associé (Qt creator) est quelque peu contraignant.
\newline

Le Java quant à lui, présente les avantages d'être plus haut niveau que le C++  et d'avoir plusieurs librairies graphiques intégrées. Suite à la réalisation de plusieurs projets en Java (POO, Graphes, Modélisation), nous nous sommes familiarisé avec son utilisation, donc il nous aurait été plus simple de l'utiliser. L'environnement de développement que nous utilisons (IntellijIdea) pour le développement Java est complet et moderne et moins contraignant que QtCreator. Par ailleurs, le fait que Java fonctionne sur une machine virtuelle permet une portabilité du programme sur différentes plateformes jugé inutile dans le cadre de ce projet.
\newline

Notre choix se portera sur le C\#. Celui-ci dispose d'un environnement de développement puissant et complet (Visual Studio). Le framework utilisé avec le C\# (.NET) est complet et haut niveau. Il permet, par exemple, une abstraction sur des points qui auraient été plus longs à développer en C++, comme par exemple la mise en réseau du programme. La création de l'interface graphique est aussi plus simple qu'avec ses deux concurrents, en effet la bibliothèque graphique (Windows Presentation Foundation) fournie avec le framework simplifie notamment la réalisation de fenêtres. Et bien que jugé peu pertinent dans le cadre de ce projet, il est toutefois intéressant de noter qu'en terme de performance le C\# se situe entre le C++ et le Java.
\newline

Le choix du C\# par rapport au Java s'est donc fait sur la flexibilité du Framework graphique fourni par C\#, mais c'est également l'opportunité d'apprendre plus en détail ce langage, très utilisé dans le monde professionnel.
\newline

En conclusion, le C\# nous permet de simplifier un bon nombre de points qu'il est peu intéressant de développer dans le cadre de ce projet. Ce gain de temps nous permettra donc de travailler plus en profondeur les points intéressants , la qualité, la robustesse, et peut-être d'arriver plus vite à la partie finale du projet : la réalisation de l'intelligence artificielle.

\newpage
\section{Annexe}
\subsection{Webographie}
\begin{itemize}
 \item \url{http://www.ffothello.org/informatique/algorithmes/}
 \item \url{http://wannabe.guru.org/scott/hobbies/chess/}
 \item \url{http://www.chessopolis.com/computer-chess/#info}
 \item \url{http://www.tckerrigan.com/Chess/TSCP/}
 \item \url{https://chessprogramming.wikispaces.com/}
 \item \url{https://fr.jeffprod.com/blog/2014/comment-programmer-un-jeu-dechecs.html}
 \item \url{https://github.com/Tazeg/JePyChess}
 \item \url{http://codes-sources.commentcamarche.net/source/50090-chess-game-core-librairie-jeu-d-echec-en-c}
 \item \url{http://imagecomputing.net/damien.rohmer/data/previous_website/documents/teaching/13_0fall_cpe/3eti_software_development_c/documents_generaux/02_presentation_projet.pdf}
 \item \url{http://khayyam.developpez.com/articles/algo/genetic/}
\end{itemize}

\end{document}
